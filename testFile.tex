\documentclass{article}
\usepackage{makeidx}\makeindex
\begin{document}
Biography of Lady Jane Grey 
(text taken from Wikipedia.org)


Lady Jane Grey (1536/1537 � 12 February 1554) was Queen regnant of England and Ireland after the death of \index{King Edward VI}King Edward VI from 10 till 19 July, 1553. Her claim to the throne derived from Edward VI's will, his "Device of the Succession", and from her descent from \index{Mary Tudor}Mary Tudor, Queen of France, which made her a great-niece of Henry VIII. Residing in the Tower of London during her short reign, she never left the premises again. Her \index{execution}execution in February 1554 was caused by her father's involvement in Wyatt's rebellion against the rule of Queen Mary.

Lady Jane Grey's rule of less than two weeks is the shortest rule of England in its history. Those historians that consider her a monarch have taken either the day of her \index{proclamation}proclamation as queen, 10 July, or that of her predecessor's death, 6 July, as the beginning. Hence her popular names of "The Nine Days' Queen"[1] or, less commonly, "The Thirteen Days' Queen". She is sometimes reckoned the first Queen regnant of England.

\index{Lady Jane}Lady Jane had an excellent humanist \index{education}education and a reputation as one of the most learned women of her day. A committed Protestant, she was posthumously regarded not only as a political victim but also as a martyr.


Early Life and Education

\index{Jane}Jane, the eldest daughter of \index{Henry Grey}Henry Grey, 1st Duke of Suffolk, and his wife, Lady Frances Brandon, was born at Bradgate Park in Leicestershire. The traditional view is that she was born around October 1537, but a recent biography has claimed that she was born earlier, on an unknown date in late 1536 or early 1537. Lady Frances was the daughter of Princess Mary, the younger sister of Henry VIII, and was thus the second cousin of Edward VI. Jane had two younger sisters, Lady Catherine Grey and Lady Mary Grey; through their mother, the three sisters were great-granddaughters of Henry VII and grandnieces of Henry VIII. Jane could claim descent twice from 15th century royal consort Elizabeth Woodville; paternally through Woodville's first husband, Sir John Grey of Groby, and maternally through her second husband King Edward IV. Jane received a comprehensive \index{education}education, and studied Latin, \index{Greek}Greek and Hebrew as well as contemporary languages. Through the teachings of her tutors, she became a committed Protestant.

\index{Jane}Jane had a difficult childhood. Even for the harsher standards of the time, Frances Brandon was an abusive, cruel, and domineering woman who felt that \index{Jane}Jane was weak and gentle, so held her under a strict disciplinary regime. Her daughter's meekness and quiet, unassuming manner irritated Frances who sought to 'harden' the child with regular \index{beatings}beatings. Devoid of a mother's love and craving affection and understanding, \index{Jane}Jane turned to books for solace and quickly mastered skills in the arts and languages. However, she felt that nothing she could do would please her parents. Speaking to a visitor, Cambridge scholar Roger Ascham, tutor to the Lady Elizabeth, she said:

"For when I am in the presence of either \index{parent!Father}Father or \index{parent!Mother}Mother, whether I speak, keep silence, sit, stand or go, eat, drink, be merry or sad, be sewing, playing, dancing, or doing anything else, I must do it as it were in such weight, measure and number, even so perfectly as \index{God}God made the world; or else I am so sharply taunted, so cruelly threatened, yes presently sometimes with pinches, nips and bobs and other ways ... that I think myself in hell."

In 1546, at less than 10 years old, \index{Jane}Jane was sent to live as the ward of 35-year-old Catherine Parr, then Queen Consort of England, who married Henry VIII in 1543. At this time, \index{Jane Smith}Jane became acquainted with her royal cousins, \index{Edward}Edward, Mary, and Elizabeth.


Contracts for Marriage

After Henry VIII died, Catherine Parr married Thomas Seymour, 1st Baron Seymour of Sudeley. Catherine died not long after the birth of her only child, \index{Mary Seymour}Mary Seymour, in late 1548, leaving the young \index{Jane}Jane once again bereft of a maternal figure. \index{Jane Smith}Jane acted as chief mourner at Catherine's funeral. \index{Jane Smith}Jane returned to her parents after Catherine Parr's death, yet Seymour showed continued interest in her, and she was again in his household for about two months when he was arrested at the end of 1548. Seymour's brother, Edward Seymour, 1st Duke of Somerset, who ruled as Lord Protector, felt threatened by Thomas' popularity with the young King Edward. Thomas Seymour was charged, among other things, with proposing \index{Jane Smith}Jane as a royal bride.

In the course of Thomas Seymour's following attainder and \index{execution}execution, Jane's father was lucky to stay largely out of trouble. After his fourth interrogation by the Privy Council, he proposed his daughter \index{Jane}Jane as a bride for the Protector's eldest son, Lord Hertford. Nothing came of this, however, and \index{Jane}Jane's next engagement, in the spring of 1553, was to Lord Guilford Dudley, the third son of John Dudley, 1st Duke of Northumberland. Her prospective father-in-law was then the most powerful man in the country. According to tradition, \index{Jane}Jane stated her preference for a single life, but her mother made her submit to the arrangement. On 21 May 1553, the couple were married at Durham House in a triple wedding, in which \index{Jane}Jane's sister Catherine was matched with the heir of the Earl of Pembroke, Lord Herbert, and another Catherine, Lord Guilford's sister, with Henry Hastings, the Earl of Huntingdon's heir.


Claim to the Throne and Accession

According to cognatic primogeniture, the Suffolks�the Brandons and, later, the Greys�comprised the junior branch of the heirs of Henry VII. The Third Succession Act restored Mary and Elizabeth to the line of succession, although the law regarded them as illegitimate. Furthermore, this Act authorised Henry VIII to alter the succession by his will. Henry's will reinforced the succession of his three children, and then declared that, should none of his three children leave heirs, the throne would pass to heirs of his younger sister, \index{Mary Tudor}Mary Tudor, who included Jane. Henry's will excluded the descendants of his elder sister Margaret Tudor, owing in part to Henry's desire to keep the English throne out of the hands of the Scots monarchs, and in part to a previous Act of Parliament of 1431 that barred foreign-born persons, including royalty, from inheriting property in England.

At the time of Edward's death, the crown would pass to Mary, the elder daughter of Henry VIII, and her male heirs. If she died without male issue, the crown would pass to Elizabeth and her male heirs. If she also died without male issue, the crown would pass to any male issue of Frances Brandon. In the absence of male children born to Frances, the crown would pass to any male children \index{Jane}Jane might have.

When Edward VI lay dying in 1553 at age 15, his Catholic half-sister Mary was still the heiress presumptive to the throne. However, \index{Edward}Edward named the (Protestant) heirs of his father's sister, \index{Mary Tudor}Mary Tudor as his successors in a will composed on his deathbed, perhaps under the persuasion of Northumberland. Edward and Northumberland knew that this effectively left the throne to Edward's cousin, Lady Jane Grey, who (like them) staunchly supported Protestantism. This may have contravened customary testatory law because \index{Edward}Edward had not reached the legal testatory age of 21. More importantly, many contemporary legal theorists believed the monarch could not contravene an act of Parliament, even in matters of the succession; Jane's claim to the throne therefore remained obviously weak. Other historians believed that the king could basically rule through divine right. Henry VII had, after all, seized the throne from Richard III on the battlefield.

Edward VI died on 6 July 1553. Four days later, Northumberland had Lady Jane Grey proclaimed Queen of England after she had taken up secure residence in the Tower of London, where English monarchs customarily resided from the time of accession until coronation. \index{Jane Smith}Jane refused to name her husband \index{Dudley}Dudley as king by letters patent and deferred to Parliament. She offered to make him Duke of Clarence instead.

Genoese merchant, Baptista Spinola, who witnessed \index{Jane}Jane's stately procession by water from Syon House to the Tower of London, describes her in these words, "This \index{Jane}Jane is very short and thin, but prettily shaped and graceful. She has small features, and a well-made \index{nose}nose, the mouth flexible and the lips red. The eyebrows are arched and darker than her hair which is nearly red. Her eyes are sparkling, and reddish brown in colour."He also noticed her freckled skin, and sharp, white teeth. On the day of her procession she wore a green velvet gown stamped in gold.

Northumberland faced a number of key tasks to consolidate his power after Edward's death. Most importantly, he had to isolate and, ideally, capture Lady Mary to prevent her from gathering support. As soon as Mary was sure of King Edward's demise, she left her residence at Hunsdon and set out to East Anglia, where she began to \index{rally}rally her supporters. Northumberland set out from London with troops on 14 July; in his absence the Privy Council switched their allegiance from \index{Jane}Jane to Mary, and proclaimed her queen in London on 19 July among great jubilation of the populace. \index{Jane Smith}Jane and her husband were imprisoned in the Gentleman Gaoler's apartments at the Tower of London. The new queen entered London in a triumphal procession on 3 August, and the Duke of Northumberland was executed on 22 August 1553. In September, Parliament declared Mary the rightful queen and denounced and revoked Jane's proclamation as that of a usurper.


Trial and \index{Execution}Execution

\index{Jane}Jane and Lord Guilford Dudley were both charged with high treason, together with two of Dudley's brothers.  Their trial, by a special commission, took place on 13 November 1553, at the Guildhall in the City of London. The commission was chaired by Sir Thomas White, Lord Mayor of London, and included Edward Stanley, 3rd Earl of Derby and John Bourchier, 2nd Earl of Bath. The two principal defendants were found guilty and sentenced to death. Jane's sentence was that she "be burned alive (the traditional English punishment for treason committed by women) on Tower Hill or beheaded as the Queen pleases." However, the imperial ambassador reported to Charles V, Holy Roman Emperor, that her life was to be spared.

The Protestant rebellion of Thomas Wyatt the younger in late January 1554 sealed Jane's fate, although she had nothing to do with it directly. Wyatt's rebellion started as a popular revolt, precipitated by planned marriage of Mary to the Roman Catholic Prince Philip, the future King Philip II of Spain. Jane's father (the Duke of Suffolk) and other nobles joined the rebellion. Charles V and his ambassadors pressed Mary to execute \index{Jane}Jane to put an end to any future focus for unrest. Five days after Wyatt's arrest, \index{Jane}Jane and Guilford were executed.

On the morning of 12 February 1554, the authorities took Guilford from his \index{rooms}rooms at the Tower of London to the public execution place at Tower Hill and there had him beheaded. A horse and cart brought his remains back to the Tower of London, past the \index{rooms}rooms where \index{Jane}Jane remained as a prisoner. \index{Jane}Jane was then taken out to Tower Green, inside the Tower of London, and beheaded in private. With few exceptions, only royalty were offered the privilege of a private \index{execution}execution; Jane's execution was conducted in private on the orders of Queen Mary, as a gesture of respect for her cousin.

According to the account of her \index{execution}execution given in the anonymous Chronicle of Queen Jane and of Two Years of Queen Mary, gave a speech upon ascending the scaffold:

�Good people, I am come hither to die, and by a law I am condemned to the same. The fact, indeed, against the Queen's highness was unlawful, and the consenting thereunto by me: but touching the procurement and desire thereof by me or on my behalf, I do wash my hands thereof in innocency, before \index{God}God, and the face of you, good \index{Christian}Christian people, this day.�

She then recited Psalm 51 (Have mercy upon me, O God) in English, and handed her gloves and \index{handkerchief}handkerchief to her maid. John Feckenham, a Roman Catholic chaplain sent by Mary who had failed to convert \index{Jane Smith}Jane, stayed with her during the \index{execution}execution. The executioner asked her forgiveness, and she gave it. She pleaded the axeman, "I pray you dispatch me quickly". Referring to her head, she asked, "Will you take it off before I lay me down?" and the axeman answered, "No, madam". She then blindfolded herself. \index{Jane}Jane had resolved to go to her death with dignity, but once blindfolded, failing to find the block with her hands, began to panic and cried, "What shall I do? Where is it?�  An unknown hand, possibly Feckenham's, then helped her find her way and retain her dignity at the end. With her head on the block, \index{Jane}Jane spoke the last words of Jesus as recounted by Luke: "Lord, into thy hands I commend my spirit!" She was then beheaded.

"The traitor-heroine of the Reformation", as historian Albert Pollard called her, was merely 16 or 17 years old at the time of her \index{execution}execution. Apparently, Frances Brandon made no attempt, pleading or otherwise, to save her daughter's life; Jane's father already awaited \index{execution}execution for his part in the Wyatt rebellion. \index{Jane}Jane and Guilford are buried in the Chapel of St Peter ad Vincula on the north side of Tower Green.

\index{Henry}Henry, Duke of Suffolk, Jane's father, was executed a week after \index{Jane}Jane, on 19 February 1554. His widow, Frances Brandon, did not make a good impression at court by marrying her Master of the Horse and chamberlain, Adrian Stokes. They married in March 1555, not as often said, three weeks after the \index{execution}execution of the Duke of Suffolk. She was fully pardoned by Mary and allowed to live at Court with her two surviving daughters.


Cultural depictions

After the violent attempted suppression of Protestantism by \index{Mary I}Mary I, \index{Jane}Jane being one of her first victims, and the succession of the Protestant Elizabeth I in 1558, \index{Jane}Jane became viewed as a Protestant martyr for centuries. The romantic tale of \index{Lady Jane}Lady Jane grew to legendary proportions in popular \index{culture}culture. 







	
	
 	 
	

1

\printindex
\end{document}